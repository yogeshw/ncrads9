\documentclass[11pt,a4paper]{report}

% Packages
\usepackage[utf8]{inputenc}
\usepackage[T1]{fontenc}
\usepackage{lmodern}
\usepackage{geometry}
\usepackage{graphicx}
\usepackage{hyperref}
\usepackage{listings}
\usepackage{xcolor}
\usepackage{fancyhdr}
\usepackage{titlesec}
\usepackage{booktabs}
\usepackage{longtable}

% Page geometry
\geometry{margin=1in}

% Hyperref setup
\hypersetup{
    colorlinks=true,
    linkcolor=blue,
    filecolor=magenta,
    urlcolor=cyan,
    pdftitle={NCRADS9 Manual},
    pdfauthor={Yogesh Wadadekar},
}

% Code listing style
\lstset{
    basicstyle=\ttfamily\small,
    backgroundcolor=\color{gray!10},
    frame=single,
    framerule=0pt,
    breaklines=true,
    columns=fullflexible,
    keepspaces=true,
}

% Header/footer
\pagestyle{fancy}
\fancyhf{}
\fancyhead[L]{\leftmark}
\fancyhead[R]{NCRADS9 Manual}
\fancyfoot[C]{\thepage}
\renewcommand{\headrulewidth}{0.4pt}
\renewcommand{\footrulewidth}{0.4pt}

% Title formatting
\titleformat{\chapter}[display]
{\normalfont\huge\bfseries}{\chaptertitlename\ \thechapter}{20pt}{\Huge}
\titlespacing*{\chapter}{0pt}{0pt}{40pt}

% Document info
\title{
    \vspace{2cm}
    {\Huge\textbf{NCRADS9 Manual}}\\[1cm]
    {\Large A FITS Image Viewer and Analysis Tool}\\[2cm]
    {\large Version 1.0}
}
\author{
    \textbf{Yogesh Wadadekar}\\[0.5cm]
    National Centre for Radio Astrophysics\\
    Tata Institute of Fundamental Research
}
\date{\today}

\begin{document}

% Title page
\maketitle
\thispagestyle{empty}
\newpage

% Table of contents
\tableofcontents
\newpage

% List of figures and tables (optional)
\listoffigures
\listoftables
\newpage

%======================================================================
\chapter{Introduction}
%======================================================================

\section{Overview}

NCRADS9 is a powerful FITS image viewer and analysis tool designed for astronomical data visualization. It provides an intuitive interface for viewing, analyzing, and annotating FITS images commonly used in astronomy.

\section{Features}

NCRADS9 offers the following key features:

\begin{itemize}
    \item Support for standard FITS formats and compressed files
    \item Multiple colormap options for data visualization
    \item Flexible scaling algorithms (linear, log, sqrt, asinh, etc.)
    \item Region creation and manipulation
    \item Statistical analysis tools
    \item World Coordinate System (WCS) support
    \item Contour overlays
    \item Cross-platform compatibility
\end{itemize}

\section{History}

NCRADS9 was developed at the National Centre for Radio Astrophysics (NCRA) to provide a modern, Python-based alternative to traditional FITS viewers while maintaining compatibility with DS9 region files and workflows.

%======================================================================
\chapter{Installation}
%======================================================================

\section{Requirements}

NCRADS9 requires the following:

\begin{itemize}
    \item Python 3.8 or later
    \item NumPy
    \item Astropy
    \item PyQt5 or PyQt6
    \item Matplotlib (optional, for additional plotting)
\end{itemize}

\section{Installation via pip}

The simplest way to install NCRADS9 is using pip:

\begin{lstlisting}[language=bash]
pip install ncrads9
\end{lstlisting}

\section{Installation from Source}

To install from source:

\begin{lstlisting}[language=bash]
git clone https://github.com/ncra/ncrads9.git
cd ncrads9
pip install -e .
\end{lstlisting}

\section{Verification}

Verify your installation by running:

\begin{lstlisting}[language=bash]
ncrads9 --version
\end{lstlisting}

%======================================================================
\chapter{User Guide}
%======================================================================

\section{Getting Started}

\subsection{Launching NCRADS9}

Launch NCRADS9 from the command line:

\begin{lstlisting}[language=bash]
ncrads9
\end{lstlisting}

To open a FITS file directly:

\begin{lstlisting}[language=bash]
ncrads9 image.fits
\end{lstlisting}

\subsection{Interface Overview}

The NCRADS9 interface consists of:

\begin{itemize}
    \item \textbf{Main Display}: The central area showing the FITS image
    \item \textbf{Menu Bar}: Access to file operations, view options, and analysis tools
    \item \textbf{Toolbar}: Quick access buttons for common operations
    \item \textbf{Status Bar}: Displays cursor position and pixel values
\end{itemize}

\section{Opening Files}

\subsection{Supported Formats}

NCRADS9 supports the following file formats:

\begin{table}[h]
\centering
\begin{tabular}{lll}
\toprule
\textbf{Format} & \textbf{Extensions} & \textbf{Description} \\
\midrule
FITS & .fits, .fit, .fts & Standard FITS format \\
Compressed FITS & .fits.gz, .fits.fz & Gzip or fpack compressed \\
\bottomrule
\end{tabular}
\caption{Supported file formats}
\end{table}

\subsection{Multi-Extension FITS}

For FITS files with multiple extensions, use the Frame menu to navigate between extensions or select specific extensions from the extension dropdown.

\section{Working with Regions}

\subsection{Creating Regions}

Available region shapes:
\begin{itemize}
    \item Circle
    \item Ellipse
    \item Box (Rectangle)
    \item Polygon
    \item Line
    \item Point
    \item Annulus
\end{itemize}

\subsection{Saving and Loading Regions}

Regions are saved in DS9-compatible format (.reg files):

\begin{lstlisting}
# Region file format: DS9 version 4.1
global color=green width=1
fk5
circle(12:30:45.2,+12:23:45.6,30")
\end{lstlisting}

\section{Colormaps and Scaling}

\subsection{Available Colormaps}

NCRADS9 provides multiple colormaps including:
\begin{itemize}
    \item Grey, Heat, Cool (Grayscale family)
    \item BB, HE, Rainbow, Standard (Astronomical)
    \item Viridis, Plasma, Inferno, Magma (Scientific)
\end{itemize}

\subsection{Scale Options}

\begin{table}[h]
\centering
\begin{tabular}{ll}
\toprule
\textbf{Scale} & \textbf{Best For} \\
\midrule
Linear & Uniform data ranges \\
Log & Large dynamic range \\
Sqrt & Moderate dynamic range \\
Asinh & Wide range with negatives \\
\bottomrule
\end{tabular}
\caption{Scale options and their use cases}
\end{table}

\section{Analysis Tools}

\subsection{Statistics}

The statistics tool computes sum, mean, median, standard deviation, min/max, and centroid for selected regions or the entire image.

\subsection{Histogram}

Display the distribution of pixel values with options for linear or logarithmic Y-axis and adjustable bins.

\subsection{Coordinate Systems}

Supported coordinate systems:
\begin{itemize}
    \item Image (pixel coordinates)
    \item Physical
    \item FK5 (J2000)
    \item FK4 (B1950)
    \item Galactic
    \item Ecliptic
\end{itemize}

%======================================================================
\chapter{API Reference}
%======================================================================

\section{Core Classes}

\subsection{NCRADS9 Class}

The main application class.

\begin{lstlisting}[language=Python]
from ncrads9 import NCRADS9

app = NCRADS9()
app.load_fits('image.fits')
app.run()
\end{lstlisting}

\subsection{FITSImage Class}

Represents a FITS image with associated metadata.

\begin{lstlisting}[language=Python]
from ncrads9.core import FITSImage

img = FITSImage('image.fits')
data = img.data
header = img.header
wcs = img.wcs
\end{lstlisting}

\section{Region Classes}

\subsection{Region Base Class}

All regions inherit from the base Region class.

\begin{lstlisting}[language=Python]
from ncrads9.regions import CircleRegion

region = CircleRegion(x=100, y=100, radius=25)
region.color = 'green'
region.width = 2
\end{lstlisting}

\section{Analysis Functions}

\subsection{Statistics}

\begin{lstlisting}[language=Python]
from ncrads9.analysis import compute_statistics

stats = compute_statistics(data, region=None)
print(stats['mean'], stats['std'])
\end{lstlisting}

\subsection{Histogram}

\begin{lstlisting}[language=Python]
from ncrads9.analysis import compute_histogram

hist, edges = compute_histogram(data, bins=100)
\end{lstlisting}

%======================================================================
% Appendices
%======================================================================

\appendix

\chapter{Keyboard Shortcuts}

\begin{longtable}{ll}
\toprule
\textbf{Shortcut} & \textbf{Action} \\
\midrule
\endhead
Ctrl+O & Open file \\
Ctrl+S & Save \\
Ctrl+Q & Quit \\
Ctrl+0 & Reset scale \\
Home & Reset view \\
F11 & Toggle fullscreen \\
Delete & Delete selected region \\
\bottomrule
\end{longtable}

\chapter{Configuration}

NCRADS9 stores configuration in \texttt{\~{}/.ncrads9/config.json}.

\begin{lstlisting}[language=json]
{
    "default_colormap": "grey",
    "default_scale": "linear",
    "region_color": "green",
    "region_width": 1
}
\end{lstlisting}

\end{document}
